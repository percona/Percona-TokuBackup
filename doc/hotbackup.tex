\documentclass[10pt]{article}
\usepackage[letterpaper,margin=1in]{geometry}
\title{Implementation of Tokutek Hot Backup}
\author{Bradley C. Kuszmaul}
\date{$Date$  $Rev$}
\begin{document}
\maketitle

The hot backup library maintains a ghostly mirror of the file system.  We have several things that mirror the unix file system:
\begin{itemize}
\item filenodes: Keeps track of the range locking for a file.  A file needs to be represented by its device and inode.  Represents a unix file or an inode.
\item directories: The mapping of names to filenodes. Represents the directory heirarchy of the file system, but only the part that involves open files.
\item description: Contains a pair $\left<\mbox{offset},\mbox{filenode}\right>$.  Represents a unix file description.
\item descriptors: Contains a pointer to a description.  For every open open file we have a descriptor.  (In principle several descriptors could point to the same description, but that won't happen until we implement \texttt{dup}).
\end{itemize}

What must happen:
\begin{itemize}
\item On \texttt{write(fd)}: 

 If no backup is running, we must still maintain the descriptoin (increment the offset).

 If backup is running, we must do the write in the destination space.
 (It's possible that the destination file hasn't been copied yet, in
 which case we have a choice: We don't have to mirror the write to the
 backup.

 To handle parallelism: we need to lock the range being written (in the filenode) before doing any of the updates.
  We need to lock the description so that we can update the offset and perform the writes.

\end{itemize}


\end{document}
